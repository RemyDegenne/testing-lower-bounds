\chapter{Testing error lower bounds}

\section{Generic lower bounds}

\begin{lemma}
  \label{lem:testing_bound_tv}
  %\lean{}
  %\leanok
  \uses{def:TV}
  Let $\mu, \nu$ be two probability measures on $\mathcal X$ and $E$ an event. Then $\mu(E) + \nu(E^c) \ge 1 - TV(\mu, \nu)$~.
\end{lemma}

\begin{proof}
\end{proof}

\begin{lemma}
  \label{lem:testing_bound_renyi_mean}
  %\lean{}
  %\leanok
  \uses{def:Renyi}
  Let $\mu, \nu$ be two probability measures on $\mathcal X$ and $E$ an event. Let $\alpha \in (0,1)$. Then
  \begin{align*}
  \mu(E)^\alpha + \nu(E^c)^{1 - \alpha}
  \ge \exp\left(-(1 - \alpha) R_{\alpha}(\mu, \nu)\right)
  \: .
  \end{align*}
\end{lemma}

\begin{proof}
\uses{lem:renyi_data_proc_event}
Let $\mu_E$ and $\nu_E$ be the two Bernoulli distributions with respective means $\mu(E)$ and $\nu(E)$.
By Lemma~\ref{lem:renyi_data_proc_event}, $R_\alpha(\mu, \nu) \ge R_\alpha(\mu_E, \nu_E)$. That divergence is
\begin{align*}
R_\alpha(\mu_E, \nu_E)
&= \frac{1}{\alpha - 1}\log \left(\mu(E)^\alpha \nu(E)^{1 - \alpha}
  + \mu(E^c)^\alpha \nu(E^c)^{1 - \alpha}\right)
\\
&\ge \frac{1}{\alpha - 1}\log \left(\mu(E)^\alpha + \nu(E^c)^{1 - \alpha}\right)
\: .
\end{align*}
\end{proof}

\begin{corollary}
  \label{cor:testing_bound_hellinger}
  %\lean{}
  %\leanok
  \uses{def:Hellinger, def:Renyi}
  Let $\mu, \nu$ be two probability measures on $\mathcal X$ and $E$ an event. Then
  \begin{align*}
  \sqrt{\mu(E)} + \sqrt{\nu(E^c)}
  \ge \exp\left(-\frac{1}{2} R_{1/2}(\mu, \nu)\right)
  = 1 - H^2(\mu, \nu)
  \: .
  \end{align*}
\end{corollary}

\begin{proof}
\uses{lem:testing_bound_renyi_mean, lem:renyi_half_eq_log_hellinger}
The inequality is an application of Lemma~\ref{lem:testing_bound_renyi_mean} for $\alpha = 1/2$. The equality is Lemma~\ref{lem:renyi_half_eq_log_hellinger}.
\end{proof}

\section{Change of measure}

\begin{lemma}
  \label{lem:llr_change_measure}
  %\lean{}
  %\leanok
  %\uses{}
  Let $\mu, \nu$ be two measures on $\mathcal X$ and let $E$ be an event on $\mathcal X$. Let $\beta \in \mathbb{R}$. Then
  \begin{align*}
  \nu(E) e^{\beta} \ge \mu(E) - \mu\left\{ \log\frac{d \mu}{d \nu} > \beta \right\} \: .
  \end{align*}
\end{lemma}

\begin{proof}
\begin{align*}
\nu(E)
&\ge \mu\left[\mathbb{I}(E) e^{- \log\frac{d \mu}{d \nu} }\right]
\\
&\ge \mu\left[\mathbb{I}\left(E \cap \left\{\log\frac{d \mu}{d \nu} \le \beta\right\}\right) e^{- \log\frac{d \mu}{d \nu} }\right]
\\
&\ge e^{- \beta}\mu\left(E \cap \left\{\log\frac{d \mu}{d \nu} \le \beta\right\}\right)
\\
&\ge e^{- \beta}\left( \mu(E) - \mu\left\{ \log\frac{d \mu}{d \nu} > \beta \right\} \right)
\: .
\end{align*}
\end{proof}

\begin{corollary}
  \label{cor:kl_change_measure}
  %\lean{}
  %\leanok
  \uses{def:KL}
  Let $\mu, \nu$ be two measures on $\mathcal X$ and let $E$ be an event on $\mathcal X$. Let $\beta \in \mathbb{R}$. Then
  \begin{align*}
  \nu(E) e^{\KL(\mu, \nu) + \beta} \ge \mu(E) - \mu\left\{ \log\frac{d \mu}{d \nu} - \KL(\mu, \nu) > \beta \right\} \: .
  \end{align*}
\end{corollary}

\begin{proof}
\uses{lem:llr_change_measure}
Use Lemma~\ref{lem:llr_change_measure} with the choice $\KL(\mu, \nu) + \beta$ for $\beta$.
\end{proof}

\begin{lemma}
  \label{lem:renyi_change_measure}
  %\lean{}
  %\leanok
  \uses{def:Renyi}
  Let $\mu, \nu$ be two measures on $\mathcal X$ and let $E$ be an event on $\mathcal X$. Let $\alpha,\beta \ge 0$. Then
  \begin{align*}
  \nu(E) e^{R_{1+\alpha}(\mu, \nu) + \beta} \ge \mu(E) - e^{-\alpha \beta} \: .
  \end{align*}
\end{lemma}

\begin{proof}
\uses{lem:llr_change_measure}
Use Lemma~\ref{lem:llr_change_measure} with the choice $R_{1+\alpha}(\mu, \nu) + \beta$ for $\beta$. Then by a Chernoff bound,
\begin{align*}
\mu\left\{ \log\frac{d \mu}{d \nu} - R_{1+\alpha}(\mu, \nu) > \beta \right\}
&= \mu\left\{ \exp\left(\alpha\log\frac{d \mu}{d \nu}\right) > \exp\left(\alpha R_{1+\alpha}(\mu, \nu) + \alpha \beta\right) \right\}
\\
&\le e^{-\alpha R_{1+\alpha}(\mu, \nu) - \alpha \beta} \mu\left[\left(\frac{d \mu}{d \nu}\right)^\alpha \right]
\end{align*}
Then $\mu\left[\left(\frac{d \mu}{d \nu}\right)^\alpha \right] = \nu\left[\left(\frac{d \mu}{d \nu}\right)^{1+\alpha} \right] = e^{\alpha R_{1+\alpha}(\mu, \nu)}$. We obtained
\begin{align*}
\mu\left\{ \log\frac{d \mu}{d \nu} - R_{1+\alpha}(\mu, \nu) > \beta \right\}
\le e^{- \alpha \beta}
\end{align*}
\end{proof}

\begin{lemma}
  \label{lem:llr_change_measure_add}
  %\lean{}
  %\leanok
  %\uses{}
  Let $\mu, \nu, \xi$ be three probability measures on $\mathcal X$ and let $E$ be an event on $\mathcal X$. Let $\beta_1, \beta_2 \in \mathbb{R}$. Then
  \begin{align*}
  \mu(E) e^{\beta_1} + \nu(E^c) e^{\beta_2} \ge 1 - \xi\left\{ \log\frac{d \xi}{d \mu} > \beta_1 \right\} - \xi\left\{ \log\frac{d \xi}{d \nu} > \beta_2 \right\} \: .
  \end{align*}
\end{lemma}

\begin{proof}
\uses{lem:llr_change_measure}
Two applications of Lemma~\ref{lem:llr_change_measure}, then sum them and use $\xi(E)+\xi(E^c) = 1$.
\end{proof}

\begin{lemma}
  \label{lem:renyi_change_measure_add}
  %\lean{}
  %\leanok
  \uses{def:Renyi}
  Let $\mu, \nu, \xi$ be three probability measures on $\mathcal X$ and let $E$ be an event on $\mathcal X$. Let $\alpha, \beta \ge 0$. Then
  \begin{align*}
  \mu(E) e^{R_{1+\alpha}(\xi, \mu) + \beta} + \nu(E^c) e^{R_{1+\alpha}(\xi, \nu) + \beta} \ge 1 - 2 e^{-\alpha \beta} \: .
  \end{align*}
\end{lemma}

\begin{proof}
\uses{lem:renyi_change_measure}
\end{proof}


\section{Lower bounds on the maximum}

\begin{lemma}
  \label{lem:testing_bound_tv_hellinger_max}
  %\lean{}
  %\leanok
  \uses{def:TV, def:Hellinger}
  Let $\mu, \nu$ be two probability measures on $\mathcal X$ and $E$ an event. Then
  \begin{align*}
  \max\{\mu(E), \nu(E^c)\}
  \ge \frac{1}{2}(1 - TV(\mu, \nu))
  \ge \frac{1}{4}(1 - H^2(\mu, \nu))^2
  \: .
  \end{align*}
\end{lemma}

\begin{proof}
\uses{lem:testing_bound_tv, cor:one_sub_hellinger_squared_le_one_sub_tv}
The first inequality is Lemma~\ref{lem:testing_bound_tv} with $\mu(E) + \nu(E^c) \le 2 \max\{\mu(E), \nu(E^c)\}$.
The second inequality is a consequence of Corollary~\ref{cor:one_sub_hellinger_squared_le_one_sub_tv}.
\end{proof}

\begin{lemma}
  \label{lem:testing_bound_renyi_max}
  %\lean{}
  %\leanok
  \uses{def:Renyi}
  Let $\mu, \nu$ be two probability measures on $\mathcal X$ and $E$ an event. Let $\alpha \in (0,1)$. Then
  \begin{align*}
  \min\{\alpha, 1 - \alpha\} \log\frac{1}{\max\{\mu(E), \nu(E^c)\}} \le (1 - \alpha) R_{\alpha}(\mu, \nu)  + \log 2 \: .
  \end{align*}
\end{lemma}

\begin{proof}
\uses{lem:testing_bound_renyi_mean}
Use Lemma~\ref{lem:testing_bound_renyi_mean} and remark that $\mu(E)^\alpha + \nu(E^c)^{1 - \alpha} \le 2\max\{\mu(E), \nu(E^c)\}^{\min\{\alpha, 1 - \alpha\}}$.
\end{proof}


\begin{lemma}
  \label{lem:testing_bound_renyi_one_add}
  %\lean{}
  %\leanok
  \uses{def:Renyi}
  Let $\mu, \nu$ be two probability measures on $\mathcal X$ and let $E$ be an event on $\mathcal X$. Let $\alpha > 0$. Then
  \begin{align*}
  \max\{\mu(E), \nu(E^c)\} \ge \frac{1}{4}\exp\left( - \inf_{\xi \in \mathcal P(\mathcal X)}\max\{R_{1+\alpha}(\xi, \mu), R_{1+\alpha}(\xi, \nu)\} - \frac{\log 4}{\alpha}\right) \: .
  \end{align*}
\end{lemma}

\begin{proof}
\uses{lem:renyi_change_measure_add}
\end{proof}


\section{Product spaces}

\begin{corollary}
  \label{cor:testing_bound_renyi_n}
  %\lean{}
  %\leanok
  \uses{def:Renyi}
  Let $\mu, \nu$ be two probability measures on $\mathcal X$ and $E$ an event of $\mathcal X^n$. Let $\alpha \in (0,1)$. Then
  \begin{align*}
  \min\{\alpha, 1 - \alpha\} \log\frac{1}{\max\{\mu^{\otimes n}(E), \nu^{\otimes n}(E^c)\}} \le (1 - \alpha) n R_{\alpha}(\mu, \nu)  + \log 2 \: .
  \end{align*}
\end{corollary}

\begin{proof}
\uses{lem:testing_bound_renyi_max, lem:renyi_prod_n}
Use Lemma~\ref{lem:renyi_prod_n} in Lemma~\ref{lem:testing_bound_renyi_max}.
\end{proof}

\begin{lemma}
  \label{lem:testing_bound_renyi_one_add_n}
  %\lean{}
  %\leanok
  \uses{def:Renyi}
  Let $\mu, \nu$ be two probability measures on $\mathcal X$, let $n \in \mathbb{N}$ and let $E$ be an event on $\mathcal X^n$. For all $\alpha > 0$,
  \begin{align*}
  \max\{\mu^{\otimes n}(E), \nu^{\otimes n}(E^c)\} \ge \frac{1}{4}\exp\left( - n \inf_{\xi \in \mathcal P(\mathcal X)}\max\{R_{1+\frac{\alpha}{\sqrt{n}}}(\xi, \mu), R_{1+\frac{\alpha}{\sqrt{n}}}(\xi, \nu)\} - \frac{\log 4}{\alpha}\sqrt{n}\right) \: .
  \end{align*}
\end{lemma}

\begin{proof}
\uses{lem:renyi_prod_n, lem:testing_bound_renyi_one_add}
Use Lemma~\ref{lem:renyi_prod_n} in Lemma~\ref{lem:testing_bound_renyi_one_add}
\end{proof}

